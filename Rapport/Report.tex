\documentclass{article}

\usepackage{geometry}

\newgeometry{
	top=2in,
	bottom=2in,
	outer=1.5in,
	inner=1.5in,
}

\author{Jon Arne \& Mats}
\title{Presentasjon av resultat fra automatisk testing av software}
\date{\today}

\begin{document}
	
\begin{center}
\begin{Huge}
Here goes the front page
\end{Huge} 
\end{center}

\pagebreak
\maketitle
The title page should include the title of your thesis, author, date/year, field of
study, and name of institution (UiS and collaborating company/institution, if any). Remember
to fill out the form ”Title page – Master’s Thesis” and insert this as page 1. 
\pagebreak

\section{Abstract}
The abstract must be understandable independently of the thesis itself. It gives a
brief presentation of the problem(s) and the work that has been carried out. Main results and
important conclusions should also be included. A good deal of work ought to be spent on
writing a good abstract because this is what most people will read. The abstract should be
short. 
\pagebreak

\tableofcontents 
This shows the chapters and subchapters with page numbers. If there are
symbols and abbreviations they can be listed after the table of contents. 
\pagebreak

\section{Acknowledgments}
If you wish to thank institutions and/or persons who have been of great
help during the project, this may be done in the acknowledgements. You may use the first-
12
Student guide for bachelor’s and master’s thesis
Faculty of Science and Technology
Decision made of the Dean June 25th 2013
person mode (“I”) in this section. The use of first person is normally avoided in the actual
report. 
\pagebreak

\section{Introduction}
The introduction could consist of several sections. First comes a short
presentation of the background for the thesis, e.g. why it is important to examine this
problem. Furthermore, you should describe what the thesis is about, what has been done and
how the report is built up. 
\linebreak
\linebreak
Testing is used in every aspect of programming. It is the key way to achieve stable and working programs. 
As tests are run for every attempted solution, their efficiency is vital to increase productivity of workers. Currently its simple to recognize if everything is as it should, but in the event of a test failure log-files often turn out thousands of lines long and result files hundreds, making precious time wasted scrolling through and searching for what exactly went wrong.
\linebreak
In this thesis we hope to define what a good result design is and discuss what is necessary to recreate a better design based on already existing result files.
\linebreak
(Vi må notere en måte at vi ikke er kilden til resultatene vi jobber med	)
\iffalse
	\subsection{Background}
	Why is it important to examine this problem?
	\linebreak
	\linebreak
	Testing is used in every aspect of programming. It is the key way to achieve stable and working programs. 
	
	As tests are run for every attempted solution, their efficiency is vital to increase productivity of workers. Currently its simple to recognize if everything is as it should, but in the event of a test failure log-files often turn out thousands of lines long and result files hundreds, making precious time wasted scrolling through and searching for what exactly went wrong.
	
	\subsection{About the thesis}
	What is the thesis about?
	\subsection{What has been done}
	What have we completed?
	\subsection{Structure}
	How is the report structured?	
\fi	
\pagebreak

\section{What makes a good presentation of test results}

Test results must be quickly understandable while still providing useful deep level information.
The design has to pull the readers attention to what he might find important, while not being annoying. 
\pagebreak

\section{Theory}
Description of the work that has already been done in this field. Normally one
evaluates alternative methods and components/equipment and argues for the choices one has
made. This section also outlines the theories, methods, models, equations, etc. that are
relevant to the thesis. Remember to tell the reader of the paper where you found the
information. Write down the sources both in the text and in the reference list. 
\pagebreak

\section{Experiments}
List the equipment you have used (for instance by tables and figures),
chemicals and experimental methods. You should also include methods for literature search,
interviews, development of methods and models etc. 

The end result has some requirements, but one of these; "To display the result in an easy-to-read
structure" is of relative acceptance.
Therefore we decided to create multiple designs in multiple filetypes and discuss pros and cons.

\pagebreak
 
\section{Results}
All results should be presented. What was the outcome of the experiments, analyses,
literature search and interviews? What kind of method or model did you arrive at?
Calculations and estimates (if any) of uncertainty should be included at this point. Use tables
(with text above) and figures (with text below). In the text all tables must be referred to with
necessary explanations. It must appear logical to the reader why these tables/figures have
been included. Reference to formulas, models and literature should be made in the theory
section. It is often more appropriate to discuss and comment on the result(s) as you move
along. In that case, the results and discussion should be dealt with in the same chapter. 
\pagebreak

\section{Discussion}
This is where you evaluate and interpret your results. Compare your findings to
other available results. Discuss possible sources of error and the implications of these. A good
thesis is recognized by a reflective discussion. The discussion must focus on significant
results and observations. Avoid making the impression that you have solved every single
detail. Be honest – do not try to cover up errors and simplifications that you have made on the
way and later found to be unfortunate. Explain instead your choice of simplification and
comment on it. 
\pagebreak

\section{References}
To avoid all suspicion of cheating by copying, it is important to quote references
correctly. This also applies to the internet when it is used as a source. 

\end{document}